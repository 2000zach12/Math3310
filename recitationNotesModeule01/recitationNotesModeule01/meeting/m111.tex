\section{Hypergame}


\begin{example}[Hypergame]
Consider only two-person games in this example (chess, tic-tac-toe,..). We call the game \emph{normal} if it terminates in a finite number of moves; for example tic-tac-toe. Now here is a hypergame; the first move of a hypergame is to choose a normal game to play. Suppose for example that we want to play a hypergame and that I am the first to move. That means that I must declare what game we are playing. For example I might say, ``Let's play chess!" in which case you make the first move in chess and we keep playing until the chess game terminates (because chess is a normal game). The first person can choose any normal game. The second player always gets to make the first move in that normal game. Those are the only rules to the hypergame.  Note that a hypergame is a normal game, because if a normal game ends in $n$ moves, then the hypergame will end in $n+1$, making it finite.  So here is the catch.  
\begin{drama}
  \Character{The Count of Monte Cristo}{you}
  \Character{The Barber of Seville}{me}

  \youspeaks: Let's play the normal game ``Hypergame''.  

  \mespeaks: Ok, let's play the normal game ``Hypergame''.  
  \youspeaks: Ok, cool, let's play the normal game ``Hypergame''.  
  \mespeaks: Yeah, let's play the normal game ``Hypergame''.  
  \youspeaks: Right, let's play the normal game ``Hypergame''.  
  \mespeaks: Fine, let's play the normal game ``Hypergame''.  
\end{drama}
Can you explain in your own words why this is a paradox?
\notes{5}
\end{example}
