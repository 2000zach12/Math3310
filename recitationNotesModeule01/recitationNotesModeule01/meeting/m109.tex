\section{Infinite Sets and Georg Cantor}
\begin{theorem}[Cantor's Theorem]
For any set $A$, the power set, $\powerset(A)$ is numerically larger than $A$.
\end{theorem}

This theorem says that  $\powerset(A)$ is numerically larger than $A$.  In particular  $\powerset(\mathbb{N})$ is numerically larger than $\mathbb{N}$.  If we keep in mind that $\powerset(A)$ is also a set, then $\powerset(\powerset(\mathbb{N}))$ is larger than $\powerset(\mathbb{N})$, and $\powerset(\powerset(\powerset(\mathbb{N})))$ is still larger, and so on.  Thus, there are infinitely many different sizes of infinite sets.

It is known that $\powerset(N)$ is the same size as the points on a line (i.e. $\mathbb{R}$) and this line is called the continuum.  This leads us to the most important question, \emph{The Continuum Hypothesis}.

\begin{idea}[The Continuum Hypothesis]
Is there a set whose size is larger than $\mathbb{N}$ but smaller than $\powerset(\mathbb{N})$? Cantor guessed {\bf no}, but so far this has not been proven, hence the ``hypothesis'' in the title.
\end{idea}

\subsection*{Some Problems with Denumerable Sets}

\begin{problem} [Is the union of denumerable sets necessarily denumerable?]
Is the union of denumerable sets necessarily denumerable?
\notes{7}
\end{problem}

\begin{problem}[Is the set of all infinite sets of natural numbers denumerable?]
Is the set of all infinite sets of natural numbers denumerable?
\notes{7}
\end{problem}

\newpage
\begin{problem}[Consdier the denumerable sequence of denumerable sets]
Consdier the denumerable sequence of denumerable sets, $D_1, D_2, D_3, \cdots$ and let $S$ be the union of all of these sets.  Is $S$ denumerable?
\notes{7}
\end{problem}

\begin{problem}[ Consider the set $S$ of all finite sequences of elements of $D$.]
Given a denumerable set $D$.  Consider the set $S$ of all finite sequences of elements of $D$.  Is $S$ denumerable?
\notes{7}
\end{problem}

\begin{problem}[Consdier the set of all infinite sequences of 1's and 0's. ]
Consdier the set of all infinite sequences of 1's and 0's.  Prove that set to be the same size as $\powerset(\mathbb{N})$.
\notes{7}
\end{problem}

\begin{problem}[Prove that every infinite set has a denumerable subset.]
Prove that every infinite set has a denumerable subset.
\notes{7}
\end{problem}

\newpage
\begin{problem}[Prove every infinite set can be put into a 1-1 correspondence with a proper subset.]
Prove that every infinite set can be put into a 1-1 correspondence with a proper subset of itself.
\notes{6}
\end{problem}

%\newpage
\subsection{A Paper on Cantor's Theorem}
\vspace{-2em}
I thought it would be fun to play with this idea.  Follow along in the video that accompanies this paper and we can try to understand Cantor's famous theorem about sets.  As you read this article, see if you can forge a connection between   Problem \ref{vampire6} and the game presented by Gueron.  You may also want to contrast our discussion of the proof of Cantor's theorm with the proof provided in the article. \cite{gueron2001}
 \fancyfoot[CE CO]{}
\fancyfoot[RE,RO]{}
\fancyfoot[LE,LO]{}
