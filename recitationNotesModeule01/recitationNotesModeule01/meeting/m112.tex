\section{Relations and Functions}

A brief word about notation.  We have comfortably been using notation for sets, like $A = \{1, 2, 3 \}$.  But we need to clarify the distinction between sets and ordered pairs.  The set  $A = \{1, 2, 3 \}$ is the same as the set  $\{2, 1, 3 \}$.  In other words, the sequence in which we write the elements doesn't matter.  If you have a basket with a banana and an apple in it, you also have a basket with an apple and a banana in it.  

For ordered pairs (or tripple or $n$-tuples), the \emph{order} matters.  We use parenthesis to denote an ordered $n$-tuple, in this case an ordered tripple, $(1, 2, 3)$.  This is different than the tripple $(2, 1, 3)$.  


The idea of a relation in mathematics is an extension of the idea of a relation in common language.  We have relations, family and friends for example.  But mathematicians have more formally defined the properties of relations.   Here are some examples to think about.

\begin{example}[Age and Parents]
My father is older than me, and I am older than my son, so my father is older than my son.  However, Danny is my father; I am my son's father.  But Danny is not my son's father. This type of relation has a particular name, which will be discussed shortly in the following paper.
\end{example}\label{age}

\begin{example}[Spouse]
I am my wife's spouse and my wife is my spouse.  However, I am my wife's husband, but my wife is not my husband.  Again, we have a property to describe when this sort of relationship holds and when if fails.
\end{example}

Relations relate one element to another.  In the case of binary relations, $x$ is related to $y$.  For example, $x$ likes $y$, or $x$ is more popular than $y$.  Some notation employed in binary relations is of the form $R(x,y)$, meaning $x$ stands in relation to $y$.  However, in past courses, we have grown accustomed to the notation $xRy$.  For example $x>y$ means $x$ is larger than $y$, where $R$ is the relation $>$.  It is weird to think of all the things we haven't actually stopped to think about before,...like $x<y$ is a realtion that could be $x\smiley y$ if we defined $\smiley$ to mean ``...has a faster car than...''.


Functions are a special type of relation.  A function is a relation where every element $x$ of set $S_1$ is related to only one element $y$ of set $S_2$.  (Often the sets are equal, $S_1=S_2$.) 
\begin{example} Let $S_1=S_2 = \mathbb{N}$.  Consider the relation $R = f$, the first prime number greater than $x\in S_1$.  We see that $8f11$ and $12f17$, or perhaps more comfortable on the eyes, $f(8)=11$  and $f(12) = 17$.  
\end{example}

\begin{example} Functions of two elements.  Let $S_1=S_2 = \mathbb{N}$.  Consider the relation $R = +$, as the sum of two elements.  We see that $f(x,y)=x+y$.  Here we are  using the notation $xRy$ as opposed to $R(x,y)$ because we probably think  $+(x,y)$ is a little bit strange.  But understand that all of this is simply notation.  They are both saying the exact same thing.  
\end{example}

\newpage
\subsection{A Paper on Relations}
This is a short presentation of the various relations most commonly encounters.  There is an interresting graph at the end of the paper where all of the 64 combinations are presented.  It is  a Venn diagram on steroids.  The proofs are really straightforward and serve as an exercise is thinking about how the types of relations affect one anothers.\cite{slonneger1977}
\fancyfoot[CE CO]{}
\fancyfoot[RE,RO]{}
\fancyfoot[LE,LO]{}
