\section{First-Order Logic}
We will now define  tableaux for \emph{First-Order Logic}.  We build on the previous eight talueaux rules of Propositional Logic along with  fours rules for the quantifiers.  Just as in the previous discussion, let's set up the tableau rules for $\forall$ and $\exists$.  There are four of these, and none of them introduce branches to the tableau.



\begin{example}[$T ~\forall~x\phi(x)$]
In the tableau this takes the form:
		\AxiomC{$T ~\forall~x\phi(x)$}
		\UnaryInfC{$T~\phi(a)$}	
		 \DisplayProof
~~where $a$ is any parameter

\vspace{1em}
\ifKey 
\color{red} If $T ~\forall~x\phi(x)$, then we can directly infer that $T \phi(a)$, where $a$ is any parameter.  Basicallys, the statement says, ``For all $x$'s, there is a $\phi(x)$, meaning that it is true for whatever $a=x$ that we choose.
 \else
\notes{5}	 
\fi    		
\end{example}

\begin{example}[$T ~\exists~x\phi(x)$]
In the tableau this takes the form:
		\AxiomC{$T ~\exists~x\phi(x)$}
		\UnaryInfC{$T~\phi(a)$}	
		 \DisplayProof
~~where $a$ is a parameter \emph{that has not yet shown up in the tableau}.

\vspace{1em}
\ifKey 
\color{red}If $F~\exists~x\phi(x)$, then we can directly infer that $F \phi(a)$, where $a$ is any parameter.  Basicallys, the statement says, ``For all $x$'s, there is a $\phi(x)$, meaning that it is true for whatever $a=x$ that we choose.
 \else
\notes{5}	 
\fi    		
\end{example}

\begin{example}[$F ~\forall~x\phi(x)$]
In the tableau this takes the form:
		\AxiomC{$F ~\forall~x\phi(x)$}
		\UnaryInfC{$F~\phi(a)$}	
		 \DisplayProof
~~where $a$ is a parameter \emph{that has not yet shown up in the tableau}.


\vspace{1em}
\ifKey 
\color{red}If $F~\forall~x\phi(x)$, then we can directly infer that $F \phi(a)$, where $a$ is any parameter.  Basicallys, the statement says, ``For all $x$'s, there is a $\phi(x)$, meaning that it is true for whatever $a=x$ that we choose.
 \else
\notes{5}	 
\fi  
		
\end{example}


\begin{example}[$F ~\exists~x\phi(x)$]
In the tableau this takes the form:
		\AxiomC{$F ~\exists~x\phi(x)$}
		\UnaryInfC{$F~\phi(a)$}	
		 \DisplayProof
~~where $a$ is any parameter

\vspace{1em}
\ifKey 
\color{red}If $F~\exists~x\phi(x)$, then we can directly infer that $F \phi(a)$, where $a$ is any parameter.  Basicallys, the statement says, ``For all $x$'s, there is a $\phi(x)$, meaning that it is true for whatever $a=x$ that we choose.
 \else
\notes{5}	 
\fi  
		
\end{example}


\newpage
Keeping things real, let's start off with an example.
\begin{example}[$\exists~x~Px \rightarrow\neg \forall~x\neg (Px \lor Qx)$]
Suppose we want to prove the formula
\[\exists~x~Px \rightarrow\neg \forall~x\neg (Px \lor Qx)\]
We begin as before with the negation, i.e assume false.  If we can prove that this statement is never false, then,...it is always true.
\[F ~\exists~x~Px \rightarrow\neg \forall~x\neg (Px \lor Qx)\]
\begin{minipage}{0.5\textwidth}
\begin{tableau}
{
to prove={F ~\exists~x~Px \rightarrow\neg \forall~x\neg (Px \lor Qx)},
close with=\ensuremath{\times}
}
[F ~\exists x~Px \rightarrow\neg \forall~x\neg (Px \lor Qx)
[T~ \exists x~Px, just= by (1)
[F~ \forall x\neg (Px \lor Qx), just= by (1)
[T~\forall x~\neg (Px\lor Qx), just =by (3)
[T~ Pa, just= by (2)
[T~\neg(Pa\lor Qa), just= by (4)
[F~Pa\lor Qa, just= by (6)
[F~Pa, just= by (7), close]]]]]]]]
\end{tableau}

And we can close because line (8) contradicts (5).
\end{minipage}
\begin{minipage}{0.5\textwidth}
\begin{enumerate}
\item  $F ~\exists x~Px \rightarrow\neg \forall~x\neg (Px \lor Qx)$
\vspace{1em}
\notes{2} 
\item $T~ \exists x~Px, just= by (1)$
\vspace{1em}
\notes{2} 
\item $F~ \forall x\neg (Px \lor Qx)$
\vspace{1em}
\notes{2} 
\item $T~\forall x~\neg (Px\lor Qx)$
\vspace{1em}
\notes{2} 
\item $T~ Pa$
\vspace{1em}
\notes{2} 
\item $T~\neg(Pa\lor Qa)$
\vspace{1em}
\notes{2} 
\item $F~Pa\lor Qa$
\vspace{1em}
\notes{2} 
\item $F~Pa$
\vspace{1em}
\notes{2} 
\end{enumerate}
\end{minipage}
\end{example}


\newpage
\begin{problem}[$(\forall x~Px\lor \exists x(Px\rightarrow Qx)) \rightarrow \exists x~Qx.$]
Prove
\[(\forall x~Px\lor \exists x(Px\rightarrow Qx)) \rightarrow \exists x~Qx\]
\begin{minipage}{0.5\textwidth}

%
%\begin{tableau}
%{
%to prove={F ~(\forall x~Px\lor \exists x(Px\rightarrow Qx)) \rightarrow \exists x~Qx},
%close with=\ensuremath{\times}
%}
%[F ~(\forall x~Px\lor \exists x(Px\rightarrow Qx)) \rightarrow \exists x~Qx
%[T~ \forall x~Px\lor \exists x(Px\rightarrow Qx), just= by (1)
%[F~  \exists x~Qx, just= by (1)
%[T~\forall x~Px, just =by (2)
%[T~ \exists  x~(Px\rightarrow Qx), just= by (2)
%[T~Pa\rightarrow Qa, just= by (5)
%[T~Pa, just= by (4)
%[F~Qa, just= by (3)[F~Pa][T~Qa, just = both by(6)] ]]]]]]]]
%\end{tableau}

\begin{tableau}
{
to prove={F ~(\forall x~Px\lor \exists x(Px\rightarrow Qx)) \rightarrow \exists x~Qx},
close with=\ensuremath{\times}
}
[F ~(\forall x~Px\lor \exists x(Px\rightarrow Qx)) \rightarrow \exists x~Qx
[T~ \underline{\hspace{10em}}, just= by (1)
[F~  \underline{\hspace{10em}}, just= by (1)
[T~ \underline{\hspace{10em}}, just =by (2)
[T~  \underline{\hspace{10em}}, just= by (2)
[T~ \underline{\hspace{10em}}, just= by (5)
[T~ \underline{\hspace{10em}}, just= by (4)
[F~ \underline{\hspace{5em}}, just= by (3)[F~Pa][T~Qa, just = both by(6)] ]]]]]]]]
\end{tableau}


And we can close because line (9) contradicts (6).


\ifKey
\vspace{3em}
\color{red}
\begin{tableau}
{
to prove={F ~(\forall x~Px\land \exists x(Px\rightarrow Qx)) \rightarrow \exists x~Qx},
close with=\ensuremath{\times}
}
[F ~(\forall x~Px\land \exists x(Px\rightarrow Qx)) \rightarrow \exists x~Qx
[T~\forall x~Px\land \exists x(Px\rightarrow Qx) , just= by (1)
[F~ \exists x~Qx , just= by (1)
[T~ \forall x~Px, just =by (2)
[T~ \exists x(Px\rightarrow Qx), just= by (2)
[T~Pa\rightarrow Qa, just= by (5)
[T~ Pa, just= by (4)
[F~Qa, just= by (3)[F~Pa,close][T~Qa, close, just = both by(6)] ]]]]]]]]
\end{tableau}

\color{black}
\fi

\end{minipage}\hfill
\begin{minipage}{0.4\textwidth}
\begin{enumerate}
\item  \label{fol1}
\vspace{1em}
\ifKey
\color{red} We begin by assuming that the propostition is false, and then the ``hope'' is that we can establish that the propostition is NEVER false, meaning it is always true, and therefor a theorem.
\color{black}
\else
\notes{2} 
\fi

\item  \label{fol2}
\vspace{1em}
\ifKey
\color{red} This follows from the tableau rules for propositional logic (the previous section).  This is the firsst part of the condition that will make an implication false.
\color{black}
\else
\notes{2} 
\fi


\item \label{fol3}
\vspace{1em}
\ifKey
\color{red}  This is the second part of the condition that will make an implication false.
\color{black}
\else
\notes{2} 
\fi


\item \label{fol4}
\vspace{1em}
\ifKey
\color{red} The $\land$ in line \ref{fol2} infers that both $xPx$ and $\exists x(Px\rightarrow Qx)$ are true. So this is the first one....
\color{black}
\else
\notes{2} 
\fi

\item \label{fol5}
\vspace{1em}
\ifKey
\color{red}and this is the second term in the  $\land$ from line \ref{fol2}. 
\color{black}
\else
\notes{2} 
\fi


\item \label{fol6}
\vspace{1em}
\ifKey
\color{red} Line \ref{fol5} says that $Px\rightarrow Qx$ holds for all $x$, so we allow $a$ to be such an $x$
\color{black}
\else
\notes{2} 
\fi

\item \label{fol7}
\vspace{1em}
\ifKey
\color{red} Line \ref{fol4} says that $Px$ holds for every $x$, so it has to hold for $Pa$.
\color{black}
\else
\notes{2} 
\fi

\item \label{fol8}
\vspace{1em}
\ifKey
\color{red} Line \ref{fol3} says tha it is false that there is any $x$ such that $Qx$, and therefore in particular, $Qa$ must be false.
\color{black}
\else
\notes{2} 
\fi

\item \label{fol9}
\vspace{1em}
\ifKey
\color{red} Line \ref{fol6} branches, forming \ref{fol9}, both terminating.
\color{black}
\else
\notes{2} 
\fi

\end{enumerate}
\end{minipage}
\end{problem}
We end our exploration of sets and logic here, but I want to leave you with a list of topics to which we have journeyed precariously close.  At this point, if you find these ideas thus far interesting, you might consider reading further into the following topics. I think you have been given the keys that open the door to get to the next level.
\begin{enumerate}
\item G\"odel's Incompletness Theorem.  \hfill Totally famous!!
\item Elementary Arithmetic.   \hfill I kid you not! How do we know it works?
\item Formal Systems.  \hfill This converts implicit (recursive) definitions into explicit.
\item Peano Arithmetic. \hfill Who said $1+1=2$?
\item The Unprovability of Consistency.  \hfill  More from G\"odel.
\end{enumerate}

\newpage
\subsection{A Paper on What Lincoln Really Meant to Say}
I just thought this was soooo nerdy that I had to include it.  Sheldon from the \emph{Big Bang} would probably dig this.
%Now that we see how the logic tables can be used to solve these riddles, the following paper takes things a step further by introducing more personality types. 
\fancyfoot[CE CO]{}
\fancyfoot[RE,RO]{}
\fancyfoot[LE,LO]{}
%------------------------------------------------------------------------------------------------------------
\begin{center}
\frame{\includegraphics[width = 4.5in]{./papers/Lincoln.pdf}}
\end{center}

\fancyfoot[CE CO]{Dr. Heavilin}
\fancyfoot[RE,RO]{Online Version}
\fancyfoot[LE LO]{Fall 2020}

