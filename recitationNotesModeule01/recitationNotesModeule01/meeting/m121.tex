\section{Introduction to First Order Logic}

Now we connect propositional logic to the notion of \emph{all} and \emph{some}.  In day-to-day speaking, the idea of ``some'' does not always imply a plural, but rather means ``at least one, not two or more.''  It means ``one or more'' in logic and mathematics.  
\begin{example}[Some people are good.]
This statement only means that there is at least one good person (Noah, for example) and everyone else is trash.
\end{example}
As for the notion of ``all'' also has a bit of a twist that comes from the idea of \emph{vacuously true}.  Vacuous truth is sort of irritating but useful.  Actually, it is more a technicality than anything else, but it helps us pin down some sticky problems.  The best way to explore this is with an example.   


\begin{example}[All Hobbits have five legs.]
This statement is true since there are no Hobbits (sorry  \frownie), and therefore all statements about Hobbits are true.
\end{example}
Now let's get straight to playing around with these ideas.
\begin{problem}[Fantasy Island.]
We are visiting another island. In speaking with \emph{all} the people on this island each one says the same thing, ``Some of use are Knights and some of us are Knaves.''  Who lives on the island?
\notes{4}
\end{problem}

\begin{problem}[Back to Fantasy Island.]
We are visiting an island where people are all knights or knaves. In speaking with the people on this island each one says the same thing, ``All of is here are of the same type.''  Who lives on the island?
\notes{4}
\end{problem}

\begin{problem}[On to another Fantasy Island.]
We are visiting another island where people are all knights or knaves. In speaking with the people on this island each one says the same thing, ``Some of is here are knights and some of us are knaves.''   Who lives on the island?
\notes{4}
\end{problem}

\begin{problem}[More Fantasy Island.]
At this island. This time we are wondering if being a vegan has something to do with begin a knight or knave.  All the people say, ``Every knight here is a vegan."   What can we conclude about the type of people on the island (knights or knaves) and can we conclude anything about the vegan issue?
\notes{4}
\end{problem}

\newpage
\begin{problem}[Even More Fantasy Island.]
..Another island.  Now  each person says, ``Some of us here are knaves and vegan."  What can we conclude?
\notes{4}
\end{problem}



\begin{problem}[Still Even More Fantasy Island.]
Now all the people here are the same, and all the people say, ``If I am a vegan, then everyone is a vegan."   What can we conclude?
\notes{4}
\end{problem}



\begin{problem}[OMG Still Even More Fantasy Island.]
Again, all the people are the same type, and each person says, ``Some of us are vegans, but I'm not."   What can we conclude?
\notes{4}
\end{problem}

\begin{problem}[Finally, the last Fantasy Island.]
Just like the last problem, all the people are the same type, and each person says something just slightly different than on the last island, ``Some of us are vegans.  I'm not."   What can we conclude?
\notes{4}
\end{problem}


\subsection{Knights and Knaves}
Now for some fun and games.  A popular logic puzzle format is structured aound determining the nature of a person who makes a statement as a truth telling knight or a consistent lying knave.  In the following problems, we assume that the persons involved are all knights or knaves.  Our job is to determine which.  
\begin{problem}[You encounter Adalbert and Phelony eating peeled grapes from a bowl made of unobtainium.]
\begin{itemize}
\item You encounter Adalbert and Phelony eating peeled grapes from a bowl made of unobtainium.  Adalbert says to you, ``Both of us are knaves.'' Which is Adalbert and which is Phelony, a knight or a knave?
\item Suppose Adalbert says, ''At least one of us in a knave."  Now what can we conclude?
\item Phelony says, ``Adalbert and I are the same type, both knights or both knaves''.  Now what?
\end{itemize}
Consider constructing truth tables! Let Adalbert be $p$ and Phelony be $q$.  Perhaps you can let knights be $\mathbf{T}$, for truth, and knaves by $\mathbf{F}$. Once we have a truth table, the truth will be revealed.
\end{problem}
%\newpage

\subsection{A Paper on Knights and Knaves}
%Now that we see how the logic tables can be used to solve these riddles, the following paper takes things a step further by introducing more personality types. 
\fancyfoot[CE CO]{}
\fancyfoot[RE,RO]{}
\fancyfoot[LE,LO]{}

%------------------------------------------------------------------------------------------------------------
\begin{center}
\frame{\includegraphics[width = 4.5in]{./papers/KnightsKnavesNormalsAndNeutrals.pdf}}
\end{center}

\fancyfoot[CE CO]{Dr. Heavilin}
\fancyfoot[RE,RO]{Online Version}
\fancyfoot[LE LO]{Fall 2020}

