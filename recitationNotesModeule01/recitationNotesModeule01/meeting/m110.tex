\section{Paradoxes}
\begin{quotation}
``A paradox is a truth standing on its head to attract attention.''

\hfill -Someone else
\end{quotation}

For a great introduction to the idea of a paradox, follow this link to a YouTube video by  Julia Galef,

\url{https://www.youtube.com/watch?v=Cbnep17Mwqg}

Just a little background, I like Galef's story.  She is a sociolgist who realized how important mathematics is and decided to get a degree in statistics.  For me she is simply evidence that people in all disiplines need to study mathematics in order to understand the complex questions facing every profession.  (end of soapbox)


Her video actually tells you how to resolve a certain paradox, and what doesn't really constitute a resultion.  For example, explaining why one agrument is illogical provides a resolution; not explaining why both are logical - because that is the reason the paradox is a paradox.  She says it is not enough to argue that one argument makes more sense to you personally, because both sides make sense to somebody.  Consdier the following example


\begin{example}[The Barber Paradox]
A male barber of a certain town shaved all men of the town who did not shave themselves, and only such men. Thus if a man of the town did not shave himself, the barber would shave him, but if a man of the town did, he did not shave him. Did the barber shave himself or didn't he? If he shaved himself, then he shave someone who shaved himself, which he was not supposed to do. If he failed to shave himself, then he failed to shave someone who didn't shave himself which violates the given conditions that he always shaves anyone who doesn't shave himself.  Thus either way we get a contradiction.  

Is there a  resolution to this paradox?
%\notes{2}
\end{example}

\begin{example}[Russell Paradox]
Call a set ordinary if it is not a member of itself, and extraordinary if it is a member of itself. Let $M$ be the set of all ordinary sets. Is $M$ ordinary or not? This is the way we get a contradiction: supposed $M$ is ordinary. Then $M$ is in $M$, which  contains all ordinary sets, but being in $M$ makes $M$ extraordinary by definition. Thus it is paradoxical to assume that $M$ is ordinary. On the other hand, suppose $M$ is extraordinary. This is a member of itself, i.e. $M$ is in $M$. But the only sets in the set $M$ are ordinary sets.  

Is there a  resolution to this paradox?
%\notes{2}
\end{example}

\begin{example}[Autological/Heterological]
Call an adjective ``autological'' if it has the property it describes and call it ``heterological" if it does not. For example the adjective``polysyllabic" is itself polysyllabic, Hense it is autological.  However, the adjective ``monosyllabic" is not monosyllabic, so it is heterlogical.  Is the adjective ``heterological'' heterological?
\end{example}

\begin{example}[The Berry Paradox]
\begin{quotation}
$$x = [\textrm{The smallest natural number not describable in less than twelve words.}]$$
\end{quotation}
That description uses only eleven words.  

%Is there a  resolution to this paradox?
%\notes{3}
\end{example}

