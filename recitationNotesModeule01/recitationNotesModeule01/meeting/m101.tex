\section{\LaTeX}
\begin{remark}
It is important to stress several ideas about \LaTeX.
\ifKey
\color{red}

\begin{itemize}
\item \LaTeX is free.
\item It is the industry standard for scientific/mathematics publishing.  Even the book for this course was written in \LaTeX.
\item It is super easy to learn, and there are loads of online resources available for help.  My favorite is \emph{StackOverflow} and I almost always go there first when I need help.
\item It is platform independent, looks beautiful, and makes you look professional. 

\item Using an online service like \emph{Overlead} makes it super efficient to collaborate on group projects.
\item Easy to learn and you can put it on your resume.
\end{itemize}
\color{black}
\else
\notes{5}
\fi

\end{remark}

\ifKey
\color{red}
\LaTeX can be found here.
\begin{itemize}
	\item Overleaf:   We will be using this in the course.  Open an account today.
	\item TeXShop: You can install \LaTeX locally on your own machine by going to the MikTek website.  Recommend installing one of the many free GUI front-ends that speed or simplify up some tasks.
\end{itemize}
\color{black}
\else
\notes{5}
\fi

At this point you should offer a brief introduction to \LaTeX, illustrating how to create a document in \emph{Overleaf}, and then produce a pdf so that students can turn in their first reflection assignemnt.  You may want to hand out a document with some starter tips, or direct them to a website.  This is where you can teach the material in whatever way you deem most instructive.  Be aware that some student have ironically had almost no practical experience with computers.  If you encounter too mant basic questions that are beneath most of the class, then redirect that student(s) to your office hours.  Do no get bogged down with one or two individals at the expense of all other students.

\subsubsection{Reflection 1}
Make the students aware of the first writing assignment.  You might even want them to start in class by downloading the .tex file from Canvas and opening it in \emph{Overleaf}.  Field questions about the assignment, but make sure to clarify the following:
\ifKey
\color{red}

\begin{itemize}
\item This assignment is designed to give you an introduction to \LaTeX, gain familiarity with the syntax and the process.  All submissions this semester will be in \LaTeX - there is no suitable substitute (i.e MS Word)
\item A reflection is a writing assignment about an idea or an article in this class.  It is not simply your own opinion.  Rather you are expected to support you opinion with context, references to the material.  
\item This is a creative and personal process.  There are no right answers.  But the instructions need to be followed. 
\item Keep the size to one side of one page for ease of grading.
\item Use the template posted to Canvas.
\end{itemize}  

\color{black}
\else
\notes{5}
\fi


\ifKey
\color{red}
\begin{definition}[Statement]
According to Wikipedia, in logic, the term \emph{statement} is understood to mean either:
\begin{itemize}
\item  a meaningful declarative sentence that is true or false, or
\item  the assertion that is made by a true or false declarative sentence.
\end{itemize}
In the latter case, a statement is distinct from a sentence in that a sentence is only one formulation of a statement, (i.e. there may be many other ways of expressing the same statement using different sentences.) 
\end{definition}
\color{black}
\else
\notes{5}
\fi



\ifKey
\color{red}
Hold a discussion about the following
\begin{problem} [Are they \emph{logically} saying the same thing.]
Consider the following two statements and ask yourself if they are \emph{logically} saying the same thing.
\begin{quotation}
``Good food isn't cheap. Cheap food isn't good.''
\end{quotation}
What do you think, and why?  There is a \emph{right} answer!
\end{problem}
\color{black}
\else
\notes{5}
\fi

