\section{Interdependence of the Logical Connectives}
Something that is curious about the logical connectives, is that they can be expressed in terms of each other.  By this I mean that we can use $\rightarrow$ and $\neg$ to define $\land$.  This will give us some greater comfort with these connectives, so together, let's start by working through some derivations of this sort. 

\begin{example}
Suppose you are stuck on a dererted island, and suddenly an alien appears.  You wish to establish a way to communicate logic with this alien.  The alien shows you and $\land$ and negation $\neg$. The alien does this my waving its tenticles and piling up sand, then pointing with a stick (I don't know, just predend it fugures out how to show you these two connectives) You need to understand ``or''  $\lor$ in terms of the alien's $\land$ and $\neg$.  Mainly because you are sharing dinner and can offer the alien either a coconut or a papaya, but you want something to eat too.

...Anyway, let's defing $\lor$ in terms of $\land$ and $\neg$.
\end{example}

\begin{problem}[Define $\lor$ in terms of $\neg$ and $\land$]
To say that $(p \lor q)$ is true, it is equivalent to saying that they are both not simultaneously false. That should help you construct the formula that is logically equivalent to $(p \lor q)$

%\notes{4}
Let's do this one together,...
%\ifKey
%\color{red}
%\hfill \begin{minipage}{0.5\textwidth}
\begin{proof}
For $p\lor q$ to be true, then one of them must be true.  In other words, BOTH are not false ($\neg p \land \neg q$). Now to say that this is not the case, simply negate ($\neg (\neg p \land \neg q)$). Therefore
$$p \lor q \equiv \neg (\neg p \land \neg q)$$
%\end{minipage}
%\color{black}
%\fi
\end{proof}
\end{problem}

Now let's keep building the logical connectives.
\begin{problem}[Define $\rightarrow$ in terms of $\neg$ and $\lor$]
Say $(p \rightarrow q)$ is true using $\neg$ and $\lor$.
\notes{6}

\ifKey
\color{red}
\hfill \begin{minipage}{0.5\textwidth}
For $p\land q$ to be true, then none them can be false.  In other words, BOTH are not false $\neg (\neg p \land \neg q$). So we have
$$p \land q \equiv \neg (\neg p \lor \neg q)$$
This can be verified by a truth table.
\end{minipage}
\color{black}
\fi
\end{problem}

\begin{problem}[Define $\rightarrow$ in terms of $\neg$ and $\land$]
Define $\rightarrow$ in terms of $\neg$ and $\land$
\notes{6}

\ifKey
\color{red}
\hfill \begin{minipage}{0.5\textwidth}
$$\neg (p \land \neg q) $$
\end{minipage}
\color{black}
\fi
\end{problem}

\begin{problem}[Define $\rightarrow$ in terms of $\neg$ and $\lor$]
Define $\rightarrow$ in terms of $\neg$ and $\lor$
\notes{6}

\ifKey
\color{red}
\hfill \begin{minipage}{0.5\textwidth}
$$\neg p \lor q $$
\end{minipage}
\color{black}
\fi
\end{problem}

%\newpage
\begin{problem}[Define $\land$ in terms of $\neg$ and $\rightarrow$]
Define $\land$ in terms of $\neg$ and $\rightarrow$
\notes{6}


\ifKey
\color{red}
\hfill \begin{minipage}{0.5\textwidth}
$$\neg(p \rightarrow \neg q) $$
\end{minipage}
\color{black}
\fi

\end{problem}

\begin{problem}[Define $\lor$ in terms of $\neg$ and $\rightarrow$]
Define $\lor$ in terms of $\neg$ and $\rightarrow$
\notes{6}


\ifKey
\color{red}
\hfill \begin{minipage}{0.5\textwidth}
$$\neg p\rightarrow q$$
\end{minipage}
\color{black}
\fi
\end{problem}


\begin{problem}[Define $\lor$ in terms of only $\rightarrow$]
Define $\lor$ in terms of only $\rightarrow$
\notes{6}


\ifKey
\color{red}
\hfill \begin{minipage}{0.5\textwidth}
$$ (p \rightarrow q )\rightarrow q$$
\end{minipage}
\color{black}
\fi
\end{problem}


\begin{problem}[Define $\equiv$ in terms of $\land$ and $\rightarrow$]
Define $\equiv$ in terms of $\land$ and $\rightarrow$
\notes{6}

\ifKey
\color{red}
\hfill \begin{minipage}{0.5\textwidth}
$$ (p \rightarrow q )\land (q \rightarrow p)$$
\end{minipage}
\color{black}
\fi

\end{problem}

\begin{problem}[Define $\equiv$ in terms of $\neg$, $\land$, and $\lor$]
Define $\equiv$ in terms of $\neg$, $\land$, and $\lor$
\notes{6}

\ifKey
\color{red}
\hfill \begin{minipage}{0.5\textwidth}
$$ (p \land q )\lor (\neg q \land \neg p)$$
\end{minipage}
\color{black}
\fi
\end{problem}

\newpage
\subsection*{Joint Denial}
We have seen that the connectives can be expressed in terms of each other, and in fact all of them can be generated from just two, $\neg$ and $\land$.  We could say that $\neg$ and $\land$ form a \emph{basis} for the connectives.  (in fact, I am going to say that, right now!)

Even more interesting is that there is a connective that, all by itself, forms all the other connectives.  The \emph{joint denial} $\downarrow$ will create all five connectives ($\land, \lor, \neg, \rightarrow$, and $\equiv$).  Joint denial means ``$p$ and $q$ are BOTH false''.  Here is the table.

\[  \begin{array}{c|c|c}
  \hline  
  p & q & p\downarrow q\\
  \hline
  \hline
 \mathbf{T} &  \mathbf{T} &  \mathbf{F}\\
  \hline
 \mathbf{T} &  \mathbf{F} &  \mathbf{F}\\
  \hline
 \mathbf{F} &  \mathbf{T} &  \mathbf{F}\\
  \hline
 \mathbf{F} &  \mathbf{F} &  \mathbf{T}\\
    \hline
    \end{array}\]
    
%  \newpage  
\begin{problem}[Derive all five connectives from $\downarrow$ (hint: start with $\neg$)]Derive all five connectives from $\downarrow$ (hint: start with $\neg$)
\notes{8}

\ifKey
\color{red}
\hfill \begin{minipage}{0.5\textwidth}
\begin{itemize}
\item Start with $\neg$ from $\downarrow$:  In a rather pendantic way of speaking, we could say, ``$p$ is false and $p$ is false.  This is $p\downarrow p$,..which is just saying $p$ is false (duh) 
$$\neg p \equiv p\downarrow p $$
\item This leads us to $land$, because both $p$ and $q$ being true is the same as NOT both $p$ and $q$ are false.  So 
$$ p\land q \equiv \neg (p\downarrow p) \equiv (p\downarrow p) \downarrow  (p\downarrow p)$$
\item Now that we have $\neg$ and $\land$, we can build all of the other connectives in terms of $\downarrow$ 
\end{itemize}
\end{minipage}
\color{black}
\fi


\end{problem}

\subsection*{Alternative Denial}
There is a second connective that can also generate all five connectives, the \emph{alternative denial}.  The alternative denial say that at least $p$ or $q$ is false, $p |  q$.  


\[  \begin{array}{c|c|c}
  \hline  
  p & q & p| q\\
  \hline
  \hline
 \mathbf{T} &  \mathbf{T} &  \mathbf{F}\\
  \hline
 \mathbf{T} &  \mathbf{F} &  \mathbf{T}\\
  \hline
 \mathbf{F} &  \mathbf{T} &  \mathbf{T}\\
  \hline
 \mathbf{F} &  \mathbf{F} &  \mathbf{T}\\
    \hline
    \end{array}\]
    
\begin{problem}[Derive all five connectives from $|$ ]
Derive all five connectives from $|$.
\notes{8}


\ifKey
\color{red}
\hfill \begin{minipage}{0.5\textwidth}
\begin{itemize}
\item Start with $\neg$ from $|$:  again in a rather pendantic way of speaking, we could say, ``$p$ is false OR $p$ is false. In other words, $p$ is false.  Now build $p \land q$ by $\neg (p|q)$.  This says, ``It is not the case that BOTH $p$ and $q$ are false.  This is the same as saying $p$ and $q$ are both true.  Similarly to the previous problem, we can now construct the entire suite of connectives.
$$\neg p \equiv p\downarrow p $$
\item This leads us to $land$, because both $p$ and $q$ being true is the same as NOT both $p$ and $q$ are false.  So 
$$ p\land q \equiv \neg (p\downarrow p) \equiv (p\downarrow p) \downarrow  (p\downarrow p)$$
\item Now that we have $\neg$ and $\land$, we can build all of the other connectives in terms of $\downarrow$ 
\end{itemize}
\end{minipage}
\color{black}
\fi
\end{problem}

\newpage
\subsection*{The Whole Enchelada}
Stop and think about the truth tables we have been making.  For propositions $p$ and $q$, each connective has a unique truth table, otherwise they would be logically equivalent.  In other words $\lor$ not logically equivalent to $\land$, and so on.   So how many possible connectives can there be for $p$ and $q$?  This is a combinatorics (counting) quesiton.   

\begin{problem}[How many connectives could there possibly be for $p$ and $q$?]
We might then ask ourselves, how many connectives could there possibly be for $p$ and $q$?  Thinking binary might help.  Often zero and one are used for true and false.  For example,
 $$ \begin{array}{c|c|c}
  \hline  
  p & q & p\land q\\
  \hline
  \hline
1& 1& 1\\
  \hline
1& 0 & 0\\
  \hline
 0 &  1 & 0\\
  \hline
 0& 0 &  0\\
  \hline
    \end{array}$$
The quesiton is then, how many four digit strings of 0 and 1 are there?  We need a ``symbol'' for every possible table entry, and every entry will have a unique combination of true and false. This is a table we can now fill in completely.

%\notes{3}
$$\begin{array}{c|c|c|c|c|c|c|c|c|c|c|c|c|c|c|c|c|c}
 && 1 & 2 & 3 & 4 & 5 & 6 & 7 & 8 & 9 & 10 & 11 & 12 & 13  & 14 & 15 & 16\\
  \hline  
p & q	 & p\land q & p\lor q	& p\rightarrow q & 	p\equiv q	&  p\downarrow q & p|q &  	p\not\rightarrow q & 	p\not\equiv q &	q\leftarrow p &q\not\leftarrow p 	&p&q&\neg p	&\neg q	&t	&f	\\
\hline
\hline
1 &1 	 &	 &	 &	 &	 &	 &	 &	 &	&	&	&&&	&	&	&\\ 
\hline 
1&0 	 &	 &	 &	 &	 &	 &	 &	 &	&	&	&&&	&	&	&\\ 
\hline
0  &1 & 	 &	 &	 &	 &	 &	 &	 &	&	&	&&&	&	&	&\\ 
\hline
0  &	0  &	 &	 &	 &	 &	 &	 &	 &	&	&	&&&	&	&	&\\ 
\hline
\end{array}$$

One way of phrasing the final six is:
\begin{enumerate}
\addtocounter{enumi}{10}
\item $p$, regardless of $q$
\item $q$, regardless of $p$
\item not $p$, regardless of $q$
\item not $q$, regardless of $p$
\item true  regardless of $p$ and $q$
\item false  regardless of $p$ and $q$
\end{enumerate}
\end{problem}

